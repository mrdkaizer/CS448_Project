\documentclass[12pt]{report}
\usepackage{minitoc}
\usepackage[english, greek]{babel}
\usepackage{balance}			                    % proper column balancing 
\usepackage{cite}			                        % well-formed numeric citations
\usepackage{epsfig}			                        % add support for `EPS' figures
\usepackage{epstopdf}			                    % automatic `EPS' to `PDF' conversion
\usepackage[LGR, T1]{fontenc}		                % enable `Type 1' fonts
\usepackage[pagebackref = false, pdftex]{hyperref}	% add support for hypertext marks
\usepackage{graphicx}			                    % add support for graphics
\usepackage{listings}			                    % add support for source code listings
\usepackage{mathptmx}			                    % use `Times' as the default font
\usepackage{subfigure}			                    % add support for `subfloats'
\usepackage{tikz}			                        % add support for custom graphics
\usetikzlibrary{calc} 
\usepackage{verbatim}	                            % better `verbatim'
\usepackage{xcolor}			                        % add support for color names
\usepackage{xspace}			                        % proper macro spacing
\usepackage{mdframed}
\usepackage{geometry}
\usepackage{tempora}
\usepackage[compact]{titlesec}

% Fix section, subsections and subsubsection size
\titleformat*{\section}{\normalsize\bfseries}
\titleformat*{\subsection}{\normalsize\bfseries}
\titleformat*{\subsubsection}{\small\bfseries}

% this spacing looks better
\frenchspacing

% Avoid word splitting (hyphenation)
\tolerance=1
\emergencystretch=\maxdimen
\hyphenpenalty=10000
\hbadness=10000

\titleformat{\chapter}[display]
{\normalfont%
    \Large% %change this size to your needs for the first line
    \bfseries}{\chaptertitlename\ \thechapter}{20pt}{%
    \large %change this size to your needs for the second line
}

\geometry{
    a4paper,
    left = 3.5cm,
    top = 2.5cm,
    right = 2.5cm,
    bottom = 3.0cm
}

% name, title, keywords
\def\pname{\textlatin{Centaur}\xspace}
\def\ptitle{Αυτοματοποιημένη Προσθήκη Ευπαθειών σε Προγράμματα Ιστού}
\def\pkeywords{keyword1, keyword2}

% comment out the following line to disable comments in text
% \newcommand{\commentsenabled}{X}
% comment commands
\ifdefined\commentsenabled
	\newcommand{\el}[1]{\textbf{\color[rgb]{1,0,0} elathan: #1}}
	\newcommand{\dk}[2]{\textbf{\color[rgb]{0,0,1} dkaizer: #1}}
\else
	\newcommand{\el}[1]{}
	\newcommand{\dk}[2]{}
\fi

\lstset{aboveskip=\baselineskip,belowskip=\baselineskip}
% listings setup
\lstset{
	backgroundcolor	= \color{white},
	basicstyle	= \scriptsize,
	breaklines	= true,
	captionpos	= b,
	commentstyle	= \color{gray}\ttfamily,
	frame		= lines, 
	language	= C++, 
	numbers		= left,
	tabsize		= 2,
	xleftmargin	= 6pt,
}

\lstdefinelanguage{JavaScript}{
  keywords={break, const, require, let, async, await, module, case, catch, continue, debugger, default, delete, do, else, finally, for, function, if, in, instanceof, new, return, switch, this, throw, try, typeof, var, void, while, with},
  morecomment=[l]{//},
  morecomment=[s]{/*}{*/},
  morestring=[b]',
  morestring=[b]",
  sensitive=true
}


% hyperref setup
\hypersetup{
	unicode,
	citecolor	= black,
	colorlinks	= true,
	filecolor	= black,
	linkcolor	= black,
	pagebackref	= true,
	pdfkeywords	= {\pkeywords},
	pdftitle	= {\ptitle},
	urlcolor	= black,
} 

% checkmark
\def\checkmark{\tikz\fill[scale=0.4](0,.35) -- (.25,0) -- (1,.7) -- (.25,.15) -- cycle;} 

% start with page numbering off
\pagenumbering{gobble}

% hacks
\newenvironment{code}{\small\verbatim}{\endverbatim\normalsize}
\newcommand{\note}[1]{\noindent\textbf{\color{red}[#1]}}
\newcommand{\commentout}[1]{}

% custom macros
\newcommand{\ie}{\emph{i.e.,}\xspace}
\newcommand{\eg}{\emph{e.g.,}\xspace}
\newcommand{\etc}{\emph{etc.}\xspace}

\linespread{1.5}
\renewcommand{\mtctitle}{}
\begin{document}

\begin{titlepage}
\begin{center}
\vspace*{1cm}

Ατομική Διπλωματική Εργασία

\vspace*{1cm}

\Large{\textbf{\MakeUppercase{\ptitle{}}}}

\vspace*{1cm}

\large{\textbf{Δημήτρης Κάιζερ}}

\vspace*{1cm}
\Large{\textbf{\MakeUppercase{ΠΑΝΕΠΙΣΤΗΜΙΟ ΚΥΠΡΟΥ}}}

\vspace*{2cm}

\includegraphics[width=0.2\textwidth]{ucy-logo.png}

\vspace*{2cm}

\Large{\textbf{\MakeUppercase{ΤΜΗΜΑ ΠΛΗΡΟΦΟΡΙΚΗΣ}}}

\vspace*{4cm}

\normalsize{Δεκέμβριος 2020}
\begin{tikzpicture}[remember picture, overlay]
  \draw[line width = 1pt] ($(current page.north west) + (1in,-1in)$) rectangle ($(current page.south east) + (-0.75in,1in)$);
\end{tikzpicture}
\end{center}
\end{titlepage}

\begin{titlepage}
\begin{center}
\LARGE{\textbf{\MakeUppercase{ΠΑΝΕΠΙΣΤΗΜΙΟ ΚΥΠΡΟΥ}}}

\Large{\textbf{\MakeUppercase{ΤΜΗΜΑ ΠΛΗΡΟΦΟΡΙΚΗΣ}}}

\vspace*{5cm}

\large{\textbf{\ptitle{} (\pname)}}

\vspace*{1cm}

\normalsize{\textbf{ΔΗΜΗΤΡΗΣ ΚΑΪΖΕΡ}}

\vspace*{5cm}

\normalsize{Επιβλέπων Καθηγητής}

\normalsize{Δρ. Hλίας Αθανασόπουλος}

\vspace*{3cm}

\normalsize{Η Ατομική Διπλωματική Εργασία υποβλήθηκε προς μερική εκπλήρωση των απαιτήσεων απόκτησης 
του πτυχίου Πληροφορικής του Τμήματος Πληροφορικής του Πανεπιστημίου Κύπρου}
\vspace*{1cm}

\begin{center}
	Δεκέμβριος 2020
\end{center}

\end{center}

\end{titlepage}


\newpage
\pagenumbering{roman}
\setcounter{page}{1}

\section*{Ευχαριστίες}
Θερμές ευχαριστίες στον επιβλέποντα Καθηγητή Δρ. Hλία Αθανασόπουλο για τη στήριξη και καθοδήγηση καθόλη τη διάρκεια εκπόνησης της παρούσης.
Θα ήθελα επίσης να ευχαριστήσω τους Ορφέα Βαν Ρόι και Μάρκο Αντώνιο Χαραλάμπους για την εξαιρετική τους συνεργασία και συμβολή.
Τέλος, θα ήθελα να ευχαριστήσω την οικογένειά μου για την έμπρακτη υποστήριξη που μου
πρόσφερε στα φοιτητικά μου χρόνια ώστε να φέρω εις πέρας όλους τους ακαδημαϊκούς μου
στόχους.
\vspace*{2cm}

\newpage
\section*{Περίληψη}

Ευάλωτες εφαρμογές έχουν γενικά μια ευρεία χρήση, είτε στην αξιολόγηση εργαλείων εντοπισμού, είτε ακόμα και για εκπαιδευτικούς σκοπούς.
To 2016 παρουσιάστηκε το \textlatin{LAVA} δημιουργώντας ουσιαστικά το πρώτο εργαλείο 
αυτόματης εισαγωγής ευπαθειών σε \textlatin{native} προγράμματα υπολογιστών.
Αφήνοντας πίσω όμως ένα μεγάλο φάσμα εφαρμογών, τις εφαρμογές ιστού, οι οποίες πλέον απολαμβάνουν και της μεγαλύτερης προσοχής. Έχοντας ως γνώμονα 
τα πιο πάνω, παρουσιάζουμε ένα εργαλείο αυτόματης εισαγωγής ευπαθειών για τις εφαρμογές ιστού. Το εργαλείο μας μπορεί να χρησιμοποιηθεί για την
αξιολόγηση των διαφόρων εργαλείων που υπάρχουν για εντοπισμό ευπαθειών και ευελπιστούμε να διαδραματίσει καθοριστικό ρόλο στην ανάπτυξή τους.

\dominitoc
\tableofcontents

\setlength{\parindent}{0em}
\setlength{\parskip}{1.0em}

% start page counting after introduction
\clearpage 

\pagenumbering{arabic}
\setcounter{page}{1}
\section{Introduction}

In recent years, fuzzing has become an essential part for discovering bugs in software. Automated software testing
or fuzzing is the process of generating or mutating inputs and feeding them to programs for the purpose of discovering
bugs. All along, there was the need for an automated way to discover bugs in software but what really sparked the
interest for fuzzing was the introduction of AFL~\cite{zalewski2015american}, the state-of-the-art fuzzer that produces feedback 
during fuzzing by leveraging instrumentation of the analyzed program. By creating this \emph{feedback loop}, fuzzers can greatly improve their performance as they  
can determine whether an input is interesting, namely it triggers a new code path, and use that input to produce other test 
cases.  

%Problem
Software testing plays a significant role in software development cycle because when vulnerabilities
are present, they can have severe consequences. By exploiting software bugs, adversaries can perform
data breaches, install malicious malware or even take complete control of a device. However, finding bugs before they 
become exploits is possible while also being a challenging task. Mainly because bugs are triggered when an unexpected input 
is given to the program, something which is difficult to fully simulate through statically written unit tests~\cite{aschermann2019nautilus}.

Additionally, while automated software testing has become an attractive field of research, it still has a long way to go, 
especially for web applications ~\cite{doupe2010johnny}. As the Internet infrastructure progresses, it can be noted that an 
increasing number of software written in native code, migrates to web applications. As a result, this attracts more attackers to 
target web applications in order to achieve their goals. Thus, there is a growing need for development of automated 
vulnerabilities scanners that target web applications as well as for automated vulnerabilities injection tools to evaluate the former.

%Others have done
Numerous fuzzers have been developed in the past few years that try to optimize the fuzzing process by proposing
various methodologies~\cite{aschermann2019redqueen,aschermann2019nautilus,stephens2016driller,rawat2017vuzzer,
osterlund2020parmesan,godefroid2005dart,godefroid2012sage}. For instance, most of the fuzzers take 
advantage of instrumentation on the source or binary level. That is, inserting code to the program in order to 
receive feedback when a code block gets triggered and try to adjust the generated inputs to improve code coverage. 
Others utilize concolic/symbolic execution in order to extract useful information about the program and use that 
information for improving the input generation process~\cite{stephens2016driller,godefroid2005dart,godefroid2012sage}. 
However, all these fuzzers are currently targeted towards finding 
vulnerabilities in native code, while web applications have received limited attention.
%The evolution of fuzzing native code, so far, has not been observed in the web-applications domain.

%Technical stuff
In this paper, we propose \pname{}, which is, to the best of our knowledge, the
first \emph{grey-box} fuzzer for web applications. Currently available fuzzers
for web applications act in a black-box fashion~\cite{doupe2010johnny}; they
just brute force the target with URLs that embed known web-attack
payloads. In contrast, \pname{} firstly instruments a web application by
%modifying its Abstract Syntax Tree(AST) by 
adding code that tracks all control
flows triggered by an input and notifies the fuzzer, accordingly. Notifications can be embedded
in the web application's HTTP response using custom headers or can be outputted to a shared file 
or memory region.
%When the instrumentation procedure is done, 
On the other hand, the fuzzer starts sending requests to the target and
analyzes the responses in order to realize any interesting requests that would
later help to improve the code coverage and as a result, trigger  
vulnerabilities nested deep in the web application's code.

Instrumentation of the analyzed program is \emph{key} to the fuzzing process,
since it allows the fuzzer to instantiate more efficient strategies for
mutating inputs and thus explore as much as possible of the code of the
application. For native applications, instrumentation is carried out at the
intermediate representation of the application's code (e.g., at the LLVM's IR),
where the source code is available, or directly to the
binary~\cite{zalewski2015american}. For web applications, instrumentation is
challenging, since (a) several different frameworks are used to realize web
applications, (b) applications are executed through a web server and (c) there
is no standard intermediate representation of web code. \pname{} applies all
instrumentation at the abstract-syntax tree layer of PHP applications and can
also instrument at the HHVM layer applications written in Hack\cite{hhvm}.
Therefore, our instrumentation can cover a significant amount of available web
code, while it is generic enough  -- labeling basic blocks, collecting
feedback, and embedding feedback using HTTP headers or other shared resources
are all entirely transparent features that are PHP/Hack agnostic -- to be applied on other
systems, assuming the underlying tools for processing code are available.

Evaluating fuzzing is another challenging task~\cite{klees2018Evaluation},
since migrating known vulnerabilities to existing software, in order to test
the capabilities of the fuzzer in finding bugs, can be a tedious
process~\cite{bug-reproduction}. Thus, for evaluating \pname{}, but also other
fuzzers for web applications, we develop a methodology for automatically
injecting bugs in web applications written in PHP.  Our methodology is inspired
by LAVA~\cite{dolan2016lava} and targets web applications instead of native
code. Injecting vulnerabilities in web code, again, is challenging, since
important tools used for analyzing native code and injecting vulnerabilities
(e.g., taint-tracking and information-flow frameworks), are not available for
web applications. To overcome this lack of available tools, our vulnerability
injection methodology leverages the instrumentation infrastructure we use for
building \pname{}, in the first place.


%Our tool is able to
%inject bugs to web applications that when triggered can lead to Reflected Cross-Site Scripting (RXSS).

%Additionally, we evaluate \pname{} using WordPress~\cite{wordpress} by injecting bugs through our automated bug injection tool. 
%Our analysis suggests that \pname{} can successfully discover most of our injected bugs.


\subsection{Contributions}
In this paper, we make the following contributions.

\begin{enumerate}

\item We design, implement and evaluate \pname{}, the first grey-box fuzzer
realized for discovering vulnerabilities in web applications.  \pname{}
applies instrumentation on the target web application for guiding the entire
fuzzing process.
Instrumentation can be applied on the AST level of PHP-based or on the HHVM
bytecode on Hack-based web applications for creating a \emph{feedback loop} and
utilizing it in order to increase code coverage. This feedback loop is
established using HTTP custom headers. Consequently, by leveraging the feedback
loop, \pname{} increases the number of potential vulnerabilities triggered.

\item We design and implement a methodology for automated bug injection in web
applications written in PHP. For injecting artificially created
vulnerabilities, we crawl \pname{}-based instrumented applications and we
insert bugs in places that can be potentially executed. Our bug-injection
methodology is \emph{not only} essential for evaluating \pname{} but also
vital for the progression of further research in vulnerability finding for
web software.

\item We thoroughly evaluate \pname{} in terms of coverage, throughput and
efficiency in finding unknown bugs.  For better understanding the measured
capabilities of \pname{} we compare our results with three existing
web-application fuzzers. \pname{} is the only fuzzer that reports coverage
information; in particular, \pname{} can cover about 21.5\% of the entire
WordPress code, which contains around \emph{half a million} LoCs, in 50 hours
of fuzzing. As expected, \pname{} is slower, in terms of throughput, due to the
involved instrumentation. In fact, another popular fuzzer, \wfuzz{}~\cite{wfuzz} is
three times faster when fuzzing Drupal, but this is something to be
expected, since the reduction of the throughput due to the instrumentation
pays off in increased coverage in the long run. Finally, \pname{}, compared to the
other three fuzzers, finds the most injected vulnerabilities (30 with the
second one being \wfuzz{} with 28) for a fuzzing session that lasts 65 hours.

%using WordPress by injecting bugs through our automated bug injection tool. Our analysis suggests that \pname{} can successfully discover most of our injected bugs.

\item To foster further research in the field, we release all of our
contributions, namely the toolchain for instrumenting PHP/Hack applications,
the actual fuzzer, and the toolchain for injecting bugs in web applications, as
open source.

\end{enumerate}

%\noindent
%We will release the code of both \pname{} and the automated bug injection tool as open source.





\input{background}
\input{lava}
\input{methodology}
\input{implementation}
\input{evaluation}
\input{related-work}
\input{conclusion}

% bibliography

\bibliographystyle{abbrv}
\balance

\selectlanguage{english}
\renewcommand{\bibname}{\textgreek{Βιβλιογραφία}}

\bibliography{thesis}


% that's all folks
\end{document}
