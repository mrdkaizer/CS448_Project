\begin{abstract}

Software testing is a promising technique for discovering unknown
vulnerabilities in programs. In particular, software testing has been realized
in the form of \emph{fuzzing} of native code, where software is exercised using
a vast amount of inputs for inferring if any of them introduces
security-related side effects. For instance, a program crash when processing a
given input may be a signal for memory-corruption vulnerability.

Although fuzzing is significantly evolved in analyzing native code, web
applications have received limited attention, so far. In this paper, we design,
implement and evaluate \pname{}, which is, to the best of our knowledge, the
first \emph{gray-box} fuzzer for web applications. \pname{}, before analyzing a
web application, carefully instruments all executing code in order to create a
\emph{feedback loop} between the fuzzer and the analyzed software. \pname{}
supports instrumenting PHP and HACK applications.

Moreover, we provide the first attempt for automatically synthesizing
reflective Cross-site Scripting (XSS) vulnerabilities in vanilla web
applications. Using an additional instrumentation pass, we can successfully
inject XSS bugs in web apps and then instruct \pname{} to discover them. We
demonstrate both bug injection and \pname{} using WordPress and Drupal. \pname{} 
can successfully discover our injected bugs faster than other black-box fuzzers.

\end{abstract}

